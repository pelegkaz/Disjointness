\begin{abstract}	
Communication complexity plays a powerful role in proving lower bounds in distributed computing, and is one of primary tools to study the protocols with limited bandwidth (CONGEST model). a common technique is to reduces problems to the well-developed theory of $2$-player communication complexity, but some of these reductions sometime fall short of capturing an $n$-player problem.Therefore a natural avenue of study is multi-party communication complexity. In this work we study multi party communication complexity, in hope to improve our understanding of these types of problems. In particular, we studied simultaneous multi-party communication complexity. 

For two player simultaneous complexity, it is known that giving the players a public (shared) random string is much more useful than private randomness: public-coin protocols can be unboundedly better than deterministic protocols, while private-coin protocols can only give a quadratic improvement on deterministic protocols.

We extend the two-player gap to multiple players, and show that the private-coin communication complexity of a $k$-player function $f$ with deterministic cost $D(f)$, has communication cost $\Omega(\sqrt{D(f)})$ for any $k \geq 2$. Perhaps surprisingly, this bound is tight: although one might expect the gap between private-coin and deterministic protocols to grow with the number of players, we show that the All-Equality function, where each player receives $n$ bits of input and the players must determine if their inputs are all the same, can be solved by a private-coin protocol with $\tilde{O}(\sqrt{nk}+k)$ bits. Since All-Equality has deterministic complexity $\Theta(nk)$, this shows that sometimes the gap scales only as the square root of the number of players, and consequently the number of bits
each player needs to send actually decreases as the number of players increases.

In Distributed, I include my part in a joint work on the Sinkless Orientation problem in the LOCAL model, a work which obtained new bounds on the Distributed Lovasz Local Lemma, as well as some additional work done in turning the LOCAL algorithm to the CONGEST model.

\end{abstract}