\section{On $\ExistsEQ$}
\label{sec:exists}

The upper bound of Section~\ref{sec:upper} reduces the $\AllEQ$ problem to a collection of 2-player $\EQ$ problems,
which can then be solved efficiently using known protocols (e.g., from~\cite{BK97}).
This works because asking whether \emph{all} inputs are equal, and finding any pair of inputs that are not equal is sufficient to conclude that the answer is ``no''. What is the situation for the $\ExistsEQ$ problem, where we ask whether there \emph{exists} a pair of inputs that are equal?
Intuitively the approach above should not help, and indeed, the complexity of $\ExistsEQ$ is higher:

\begin{theorem}
	If $k \leq 2^{n-1}$, then $\Rsim_\epsilon(\ExistsEQ_{k,n}) = \Omega(k\sqrt{n})$ for any constant $\epsilon < 1/2$.
	\label{thm:ExistsEQ}
\end{theorem}
\begin{proof}
	We show that each player $i$ must send $\Omega(\sqrt{n})$ bits in the worst case,
	and the bound then follows by Observation~\ref{obs:sum}. The proof is by reduction from 2-player $\EQ_{n-1}$ (we assume that $n \geq 2$).

	Let $\Pi$ be a private-coin simultaneous protocol for $\ExistsEQ_{k,n}$ with error $\epsilon < 1/2$.
	Consider player 1 (the proof for the other players is similar).
	Assign to each player $i \in \set{3,\ldots,k}$ a unique label $b_i \in \set{0,1}^{n-1}$ (this is possible because $k \leq 2^{n-1}$.

	We construct a 2-player simultaneous protocol $\Pi'$ for $\EQ_{n-1}$ with error $\epsilon < 1/2$ as follows:
	on inputs $(x,y) \in \set{0,1}^{n-1}$,
	Alice plays the role of player 1 in $\Pi$, feeding it the input $1x$ (that is, the $n$-bit vector obtained by prepending '1' to $x$);
	Bob plays the role of player 2 with input $1y$;
	and the referee in $\Pi'$ plays the role of all the other players and the referee in $\Pi$,
	feeding each player $i$ the input $0 b_i$, where $b_i$ is the unique label assigned to player $i$.

	After receiving messages from Alice and Bob, and sampling the messages players $3,\ldots,k$
	would send in $\Pi$ when given the inputs described above,
	the referee computes the output of $\Pi$ and that is also the output of $\Pi'$.

	Because we prefixed the true inputs $x,y$ with 1, and players $3,\ldots,k$
	received inputs beginning with 0, we have $\ExistsEQ(1x, 1y, 0b_3,\ldots, 0b_k) = \EQ(x,y)$.
	Therefore $\Pi'$ succeeds whenever $\Pi$ does, and in particular it has error at most $\epsilon$.
	By the lower bound of~\cite{BK97}, then, player 1 must send $\Omega(\sqrt{n})$ bits in the worst case.
\end{proof}
We note that if $k \geq 2^n$ then $\ExistsEQ$ is trivial, as there must always exist two players with the same input.
(The depencence of $k$ on $n$ in Theorem~\ref{thm:ExistsEQ} can be improved to $k \leq (1-o(1))\cdot2^n$ by assigning inputs to player $3,\ldots,k$ more cleverly.)

The lower bound above is tight up to $O(\log k)$, and indeed we can state something more general:
for any 2-player
function $f : \mathcal{X} \times \mathcal{Y} \rightarrow \set{0,1}$,
let $\exists_k f$ be the $k$-player function that outputs 1 on input $(x_1,\ldots,x_k)$ iff for some $i,j \in [k]$ we have $f(x_i,x_j) = 1$.

\begin{lemma}
	For any 2-player function $f$ and $k \geq 2$ we have $\Rsim_{k^2 \epsilon} (\exists_k f) = O(k \cdot \Rsim_\epsilon(f))$.
	\label{lemma:exists}
\end{lemma}
\begin{proof}
	Let $\Pi = (\Pi_A, \Pi_B, O)$ be a 2-player private-coin simultaneous protocol for $f$ with communication complexity $\Rsim_\epsilon(f)$.

	We construct a protocol $\Pi'$ for $\exists_k f$ as follows:
	on input $(x_1,\ldots,x_k)$, each player $i$ samples 
	two messages, $M_A^i \sim \Pi_A(x_i)$ and $M_B^i \sim \Pi_B(x_i)$,
	and sends them to the referee. The referee
	samples outputs $Z_{i,j} \sim O( M_A^i, M_B^j )$ for each $(i,j) \in [k]^2$ ($i\neq j$)
	and
	outputs ``1'' iff for some pair $Z_{i,j} = 1$.

	If $\exists_k f(x_1,\ldots,x_k) = 1$, then there exist $i,j$ such that $f(x_i, x_j) = 1$, and for this pair of players
	we have
		$\Pr\left[ Z_{i,j} = 0 \right] \leq \epsilon$.
	Therefore the referee outputs ``1'' except with probability $\epsilon$.

	On the other hand, if $\exists_k f(x_1,\ldots,x_k) = 0$,
	then for every pair $i,j$ of players we have $f(x_i, x_j) = 0$,
	so
		$\Pr\left[ Z_{i,j} = 1 \right] \leq \epsilon$.
		By union bound, in this case the probability that the referee outputs ``1'' is bounded by ${k \choose 2} \epsilon < k^2 \epsilon$.
\end{proof}


To handle the increased error of the protocol for $\exists_k f$, we can use a protocol for $f$ that has error $O(1/k^2)$;
this is achieved by taking a constant-error protocol for $f$, executing it $O(\log k)$ independent times, and taking the majority output~\cite{KN96}.
We obtain the following:

\begin{theorem}
	For any 2-player function $f$, $k \geq 2$, and constant $\epsilon < 1/2$
	we have $\Rsim_{\epsilon} (\exists_k f) = O(k \log k \cdot \Rsim_\epsilon(f))$.
	\label{thm:exists}
\end{theorem}
\begin{corollary}
	For $\ExistsEQ$ we have
	$\Rsim_{\epsilon}(\ExistsEQ) = O( k \log k \sqrt{n})$,
	matching our lower bound up to a logarithmic factor.
\end{corollary}
