\documentclass{article}
\usepackage[utf8]{inputenc}
\usepackage{amsmath}
\usepackage[margin=1in]{geometry}
\usepackage{amsthm}
\usepackage{amsmath}
\usepackage{amssymb}
\usepackage{times}
\usepackage{paralist}
\usepackage[usenames]{color}
\usepackage{enumerate}
\usepackage{graphicx}
\usepackage{appendix}
%\usepackage[boxed,vlined]{algorithm2e}
\usepackage{lscape}
\usepackage{rotating}
\usepackage{placeins}
\usepackage{authblk}
\usepackage[normalem]{ulem}
\usepackage[ruled,boxed,vlined]{algorithm2e}
\usepackage{float}
\usepackage[colorlinks,urlcolor=black,citecolor=black,linkcolor=black]{hyperref}

\title{Disjointness CC}
\author{Peleg Kazaz}
\date{August 2019}

\begin{document}

\maketitle


%Special macros
\newcommand{\fnstyle}[1]{\mathsf{#1}}

\theoremstyle{plain}
\newtheorem{theorem}{Theorem}[section]
\newtheorem{claim}{Claim}
\newtheorem{proposition}[theorem]{Proposition}
\newtheorem{lemma}[theorem]{Lemma}
\newtheorem{observation}[theorem]{Observation}
\newtheorem{property}{Property}
\newtheorem{corollary}[theorem]{Corollary}
\newtheorem{conjecture}[theorem]{Conjecture}
\newtheorem{definition}{Definition}
\newtheorem{invariant}{Invariant}
\newtheorem*{remark}{Remark}

\renewcommand{\include}{\input}

\section{Introduction}
We are going to consider a sequential point of view for the disjointness problem for k players. Let us imagine a process in which the players go one after another and intersect their sets with the result. Starting with the full set ($[n]$), after the last player plays, we end up with the intersection of all sets. Therefore our question is whether this set is empty or not. Moreover, if the last set is empty, one of the players must subtract a large amount of elements in this process. After he plays we get a set which is in smaller order of magnitude than the one before. The difference between those sets are included in the player's zeros. That is where we have an option to learn them using small amount of communication. 
\subsection{Related Work}
\section{Preliminaries}
\subsection{Notations}
We use the popular notation: We are going to have $k$ players. For $i \in [k], X_i \in \{0,1\}^{n}$ - the $i$'th player's input.
Let us define the following sets: $A_0 = [n]$, for $i \in [k]$ : $A_i = \cap^{i}_{j=1}X_i$. Using this notation, disjointness problem is whether $A_k = \emptyset$ or not. 
\subsection{Properties}
\begin{claim}
Let $i \in [k]$ be the minimal index such that $\frac{|A_{i}|}{|A_{i-1}|} < \frac{1}{n^{1/k}}$. We assume such one exists and name it "the critical index". So for every $j < i$, $|A_j| \geq n^{1-\frac{j}{k}}$
\end{claim}
\begin{proof}
By induction of $j$. Induction basis is for $j = 0$, $A_0 = n$ by definition. The induction step: $j < i$ so $\frac{|A_{j}|}{|A_{j-1}|} \geq \frac{1}{n^{1/k}}$. By induction assumption $|A_{j}| \geq \frac{n^{1-\frac{j-1}{k}}}{n^{1/k}} = n^{1-\frac{j}{k}}$ as needed.
\end{proof}
\begin{claim}
When the sets are disjoint, there is a critical index.
\end{claim}
\begin{proof}
Since the sets are disjoint $|A_k| = 0$. Therefore, $\frac{|A_k|}{|A_{k-1}|} = 0$. So $i=k$ is such (if there is no smaller index).
\end{proof}
\section{$O(kn^{1-1/k})$ Protocol}
\subsection{Fun Facts}
\begin{equation*}
 \frac{|A_k|}{|A_0|} = \frac{|A_k|}{|A_{k-1}|}\cdot\frac{|A_{k-1}|}{|A_{k-2}|}\cdot...\cdot\frac{|A_2|}{|A_1|}\cdot\frac{|A_1|}{|A_0|} \geq \left( \min_{j \in [k]}{  \frac{|A_{j}|}{|A_{j-1}|}}\right )^k 
\end{equation*}
\begin{equation*}
 \text{DISJ} = \biguplus_{i=1}^{k}\{DISJ \land \text{i is the critical index} \} 
\end{equation*}
\begin{equation*}
    \epsilon \leq Pr[\text{DISJ}] = \sum\limits_{i=1}^k{Pr[\text{DISJ} \land \text{i is the critical index}]} \leq k \cdot \max_{1\leq i \leq k}{Pr[\text{DISJ} \land \text{i is the critical index}]} 
\end{equation*}
\begin{equation*}
    \frac{\epsilon}{k} \leq \max_{1\leq i \leq k}{Pr[\text{DISJ} \land \text{i is the critical index}]}
\end{equation*}
\subsection{The Protocol}
Our protocol is divided into rounds. In each round we find out some zeros of some player and then we ignore these indexes from now on and make our universe smaller. \newline
Every round works like this: \newline
The coordinator asks the players one by one whether $\Pr[DISJ \land \text{i is critical} | X_i = x_i \land \text{We got this far in the protocol}] \geq \frac{\epsilon}{k}$. \newline
If all of them answer negative - declare intersection. \newline
Otherwise, choose the first one to take out indexes by the following protocol: \newline
First the coordinator sends his identity ($i$) to all of the players. \newline
Now all of them parse the public random coins as samples of $A_{i-1}$ (Every player knows to parse it). Player $i$ finds a sample in which he is critical. Then he sends its index to the coordinator and after that he sends $A_i (=A_{i-1} \cap x_i)$ index by index. The coordinator sends this information to every other player.
Note that knowing this information they all can deduce $A_i \setminus A_{i-1} \subseteq X^{c}_i$ which are zeros of player $i$. They all take these zeros out (making the universe smaller $U := U \cap (A_i \setminus A_{i-1})$). If $|U| \leq n^{1-1/k}$, everyone sends their inputs to the coordinator which calculate the answer and declares accordingly.
\subsection{Analysis}
\paragraph{Error Analysis}
For each leaf $L$ in our protocol tree (determined by X,Y), we are going to argue that we error for at most $\epsilon$. Therefore in overall we don't error more than $\epsilon$. (We consider the tree given specific random coins) \newline
A leaf $L$ is a round in which we stop. \newline
The only times we error is when we answer due to the probability calculations. The other leaves have 0 error.\newline
$\Pr[ERROR | \text{Stopped at L}] = \Pr[ERROR \land DISJ | \text{Stopped at L}] = \sum\limits_{i=1}^k \Pr[ERROR \land \text{DISJ} \land \text{i is critical}| \text{Stopped at L}] = \sum\limits_{i=1}^k \mathop{\mathbb{E}}_{X_i=x_i}\Pr[ERROR \land \text{DISJ}\land \text{i is critical}| \text{Stopped at L} \land X_i=x_i] \leq \sum\limits_{i=1}^k \mathop{\mathbb{E}}_{X_i=x_i}\frac{\epsilon}{k} = \sum\limits_{i=1}^k \frac{\epsilon}{k} = \epsilon$ \newline
\newline
$\Pr[ERROR] = \mathop{\mathbb{E}}_{\text{L leaf}}[\left \Pr[ERROR | \text{Stopped at L}] \right] \leq \epsilon $
\paragraph{Communication Analysis}
\subsection{Extra}
\paragraph{Binary Distribution}
Let us consider a specific interesting distribution for $n = 2^{k-1}$. We think of $X_i \subseteq \{0, 1, ... , n-1\} $ \newline
For $i \in [k-1]$
  \[
    X_i=\left\{
                \begin{array}{ll}
                  \{0 \leq m \leq n-1 \| m_{i-1} = 0\} \text{ w.p 0.5} \\
                  \{0 \leq m \leq n-1 \| m_{i-1} = 1\} \text{ w.p 0.5}
                \end{array}
              \right.
  \]
and for $i = k$
  \[
    X_k=\left\{
                \begin{array}{ll}
                  \{0 \leq m \leq n-1 \| \underset{i=1}{\overset{k-1}{\oplus}} m_i = 0\} \text{ w.p 0.5} \\
                  \{0 \leq m \leq n-1 \| \underset{i=1}{\overset{k-1}{\oplus}} m_i = 1\} \text{ w.p 0.5}
                \end{array}
              \right.
  \]
where $m_i$ is the $i$'th bit of m in binary representation. ($m_i \mathrel{\mathop:}= ( m \mathop{\&} 2^{i-1} ) \gg i-1 $). \\
Let's pay attention for some simple properties. First of all $\forall_i |X_i| = \frac{n}{2}$. Moreover, it doesn't matter if we permute the players, for $i < k, |A_i| = 0.5|A_{i-1}|$ and generally for $i < k, |A_i| = 2^{k-1-i}$. $\Pr[DISJ] = 0.5$. The thing is that this distribution has a little entropy ($k$). 
\subsection{TODO}
Communcation analysis \newline
Tough (est?) distribution\newline
How does this protocol end \newline
Does it help to permutate the players? \newline
Do we have to tell every player everything? maybe we can get rid of the k factor? \newline
\end{document}
