\documentclass{article}
\usepackage[utf8]{inputenc}
\usepackage{amsmath}
\usepackage[margin=1in]{geometry}
\usepackage{amsthm}
\usepackage{amsmath}
\usepackage{amssymb}
\usepackage{mathtools}
\usepackage{bbm}
\usepackage{times}
\usepackage{paralist}
\usepackage[usenames]{color}
\usepackage{enumerate}
\usepackage{graphicx}
\usepackage{appendix}
\usepackage{lscape}
\usepackage{rotating}
\usepackage{placeins}
\usepackage{authblk}
\usepackage[normalem]{ulem}
\usepackage[ruled,boxed,vlined]{algorithm2e}
\usepackage{float}
\usepackage[colorlinks,urlcolor=black,citecolor=black,linkcolor=black]{hyperref}

\title{Multiparty Disjointness on Product Distributions}
\author{Peleg Kazaz}
\date{June 2020}


%General math style
\DeclarePairedDelimiter\floor{\lfloor}{\rfloor}
\newcommand{\fnstyle}[1]{\mathsf{#1}}
\newcommand{\set}[1]{\left\{#1\right\}}
\newcommand{\coloneq}{:=}
\newcommand{\st}{\medspace \middle| \medspace}
\newcommand{\reals}{\mathbb{R}}
\newcommand{\nat}{\mathbb{N}}
\newcommand{\eps}{\epsilon}
\newcommand{\norm}[1]{\left\| #1 \right\|}
\newcommand{\prob}[1]{\ensuremath{\text{\textsc{#1}}}}
\newcommand\numberthis{\addtocounter{equation}{1}\tag{\theequation}}
\DeclareMathOperator{\poly}{poly}

\newcommand{\Dfrac}[2]{%
  \ooalign{%
    $\genfrac{}{}{1.2pt}1{#1}{#2}$\cr%
    $\color{white}\genfrac{}{}{.4pt}1{\phantom{#1}}{\phantom{#2}}$}%
}

%Comments, TODO, etc.
\newcommand{\hide}[1]{ }
\newcommand{\note}[1]{ { \color{blue} #1 } }
\newcommand{\Rnote}[1]{ { \color{magenta} #1 } }
\newcommand{\TODO}[1]{ {\color{red} #1 }}

%Probability
\newcommand{\Ber}{\mathsf{B}}
\DeclareMathOperator*{\E}{E}
\DeclareMathOperator*{\Var}{Var}
\DeclareMathOperator*{\Cov}{Cov}
\newcommand{\given}{\medspace \middle| \medspace}
\newcommand{\rv}[1]{\mathbf{#1}}

%Communication & information
\newcommand{\CC}{\mathrm{CC}}
\DeclareMathOperator*{\MI}{I}
\DeclareMathOperator*{\CIC}{CIC}
\DeclareMathOperator{\HH}{H} 


\theoremstyle{plain}
\newtheorem{theorem}{Theorem}[section]
\newtheorem{claim}{Claim}
\newtheorem{proposition}[theorem]{Proposition}
\newtheorem{lemma}[theorem]{Lemma}
\newtheorem{observation}[theorem]{Observation}
\newtheorem{property}{Property}
\newtheorem{corollary}[theorem]{Corollary}
\newtheorem{conjecture}[theorem]{Conjecture}
\newtheorem{definition}{Definition}
\newtheorem{invariant}{Invariant}
\newtheorem*{remark}{Remark}

\renewcommand{\include}{\input}

\begin{document}

\maketitle

%\begin{algorithm}[H]
  %\SetAlgoLined
  %initialization\;
  %\While{not at end of this document}
  %{
    %read current\;
    %\eIf{understand}{go to next section\;current section becomes this one\;}
    %{go back to the beginning of current section\;}
  %}
  %\caption{How to write algorithms}
%\end{algorithm}

%In each iteration $r$, the coordinator asks each of the $k$ players whether or not they are significant.
%If no player is significant, the coordinator halts and outputs ``not disjoint''.
%If there is some significant player, the coordinator chooses the first player $i$ that is significant,
%and sends the index $i$ to all players.
%
%Next, using public randomness, the coordinator and the players sample infinitely many \TODO{(figure out how many really)}
%iid inputs $(X_1, \ldots , X_k)^1, \ldots (X_1, \ldots , X_k)^2,\ldots \sim \mu((\rv{X_1}, \ldots , \rv{X_k}))$.
%Player $i$ finds the first index $j$ such that he is critical in $(X_1, \ldots X_{i-1}, x_i, X_{i+1}, \ldots , X_k)^j$,
%sends $j$ to the coordinator, and then
%
%and uses 
%We use this sample in order to find elements in $[n] \setminus x_i$ as following: Critical property claims that $|A_{i-1}^j \cap x_i| / |A_{i-1}^j| < 1/n_{r-1}^{1/k}$. Now we split $A_{i-1}^j$ to sets of size $\frac{n_{r-1}^{1/k}}{2}$ by order denoted $Z^{r}_1, \ldots , Z^{r}_m$. .By counting argument, one of them doesn't intersect with $A_{i-1}^j \cap x_i$ - let us denote its index by $l$. Player $i$ sends the index $j$ to the coordinator, along with the index $l$ for the right $Z^{r}_l$ to the coordinator.
%The coordinator disseminates $j, l$ to all the other players.
%
%Observe that $Z^{r}_l \subseteq \overline{X_i}$,
%so the elements of $Z^{r}_l$ can now be removed from the universe:
%\begin{equation*}
  %U \leftarrow U_{r-1} \setminus (Z^{r}_l).
%\end{equation*}

%If $n_r \leq n^{1-\frac{1}{k}}$, every player sends his input to the coordinator which calculates the disjointness and sends the output to every player.

%\paragraph{Clarifications}
%We strongly use the fact that this is a product distribution where everyone can sample the inputs and parse them and the critical player's input doesn't affect the distribution. \newline
%For specific inputs, one of the players must be critical if there is an intersection (as proved in the lemma) but it is not easy to know which one of them is critical. 
%\TODO{up to here}





\subsection{TODO}
Add a universe set U. and pseudo code. 
Tough (est?) distribution\newline
Make sure to rethink how does this protocol end \newline
Does it help to permutate the players? \newline
Do we have to tell every player everything? maybe we can get rid of the k factor? I need to ask which one of them may have a critical index... \newline
\end{document}
